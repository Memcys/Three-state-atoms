\documentclass{book}

\usepackage{amsmath}
\usepackage[normalem]{ulem}	% \uline in substitution for \underline

\author{Bruce W. Shore}
\title{The Theory of Coherent Atomic Excitation\\Volume 2\\Multilevel Atoms and Incoherence}

\setcounter{chapter}{13}	% Chapter 13

\begin{document}

\noindent
approximation: it gives the equation
\begin{equation}
	\frac{\mathrm d}{\mathrm d t}C_1 = - i \Delta _1 C_1 - \frac{i}{2} \Omega _1^* C_2
\end{equation}
where the coefficients are
\begin{equation}
	\hbar \Delta _1 = E_1 - \hbar \dot{ \zeta _1 }
\end{equation}
\begin{equation}
\label{eq:13.1-7b}
	\hbar \Omega _1^* = - \mathrm d_12 \cdot \epsilon _1^* \mathcal{E}_1^*\
	\qquad or \qquad \hbar \Omega _1 = - \mathrm{d_21} \cdot \epsilon _1 \mathcal{E}_1.
\end{equation}
The appearance in this equation of $\Omega _1^*$, rather than $\Omega _1$, follows the convention of pairing $\Omega$ with the positive-frequency amplitude $\mathcal{E}$ and pairing $\Omega^*$ with $\mathcal{E}^*$.

Phase choice (13.1-4) %{need to insert \ref}
is suitable only when $E_1 < E_2$. If, instead, state 2 has the lower energy of this pair, then we must choose
\begin{equation}
	\dot{\zeta _2} = \dot{\zeta_1} - \omega _1,
\end{equation}
from which we obtain the equation
\begin{equation}
	\frac{\mathrm d}{\mathrm d t} C_1 = - i \Delta _1 C_1 - \frac{i}{2} \Omega _1' C_2.
\end{equation}
The Rabi frequency $\Omega _1' $that appears here differs from that of Eqn. \ref{eq:13.1-7b} by complex conjugation of the field [cf. Eqn. (10.1-22)]:
\begin{equation}
	\hbar \Omega ' = - \mathrm{d_12} \cdot \epsilon _1 \mathcal{E}_1.
\end{equation}
We may note in passing that, just as in the two-state RWA, we are at liberty to add to $ \zeta _2 $ a time-independent constant such that $ \Omega _1 $is real and non-negative at some instant of time. This additional phase does not affect the coefficients nor does it alter the RWA. If the phase of $ \mathcal{E}_1 $ remains constant, $\Omega _1 $remains real. However, when we allow phase variations (as we must in treating propagation or fluctuations) then we cannot necessarily assume the Rabi frequencies remain real at all times.

\bigskip\noindent
\textbf{The Second Equation.} The second equation, for the time derivative of $ C_2 $, has on its right hand side four exponential terms as coefficients of $ C_1 $ coefficients are those considered above; the $ C_3 $arguments are
\iffalse
\begin{itemize}
	\item $ ( \zeta _3 - \zeta _2 + \omega _1 t ) $,
	\item $ ( \zeta _3 - \zeta _2 + \omega _2 t ) $,
	\item $ ( \zeta _3 - \zeta _2 - \omega _1 t ) $,
	\item $ ( \zeta _3 - \zeta _2 - \omega _2 t ) $.
\end{itemize}
\fi
\begin{align*}
	& ( \zeta _3 - \zeta _2 + \omega _1 t ),
	& ( \zeta _3 - \zeta _2 + \omega _2 t ),\\
	& ( \zeta _3 - \zeta _2 - \omega _1 t ),
	& ( \zeta _3 - \zeta _2 - \omega _2 t ).
\end{align*}
Again a suitable choice of phases can eliminate time variation from one exponential, leaving terms that oscillate rapidly and so average to zero in the RWA. The choice depends upon whether $ E_3 $ lies above or below $ E_3 $. If $ E_2 < E_3 $then we choose
\begin{equation}
\dot{ \zeta _3 } = \dot{ \zeta _2 } + \omega _2
\end{equation}
so that the exponential arguments become
\begin{align*}
	& ( \omega _2 + \omega _1 ) t,
	& ( \omega _2 - \omega _1 ) t,
	& 2 \omega _2 t,
\end{align*}
Upon replacing the exponentials with their time averages (zero or unity) we obtain the second-step RWA equation (under the assumption $ E_1 < E_2 $)
\begin{equation}\label{eq:13.1-12}
	\frac{\mathrm d}{\mathrm d t} C_2 = - i \Delta _2 C_2 - \frac{i}{2} [ \Omega _1 C_1 + \Omega _2^* C_3]
\end{equation}
where the coefficients are
\begin{equation}
	\hbar \Delta _2 = E_2  - E_1 - \hbar \omega _1 + \hbar \Delta _1
\end{equation}
\begin{equation}
\label{eq:13.1-13b}
	\hbar \Omega _2 * = - \mathrm {d_{23}} \cdot \epsilon _2 * \mathcal{E}_2 *\
	\qquad or \qquad \hbar \Omega _2 = - \mathrm{d_{32}} \cdot \epsilon _2 \mathrm{E}_2.
\end{equation}
This expression for $ \Delta _2 $ applies when $ E_1 < E_2 $. When the energy ranking is instead $ E_1 > E_2 $the diagonal element appears as
\begin{equation}
	\hbar \Delta _2 = \hbar \omega _1 - ( E_1 - E_2 ) + \hbar \Delta _1
\end{equation}
and the coefficient of $ C_1 $ becomes $ ( \Omega _1 ')* $. If $ E_3 < E_2 $ then the proper choice of phases is
\begin{equation}
	\dot{ \zeta _3 } = \dot{ \zeta _2 } - \omega _2
\end{equation}
with corresponding modification of the formula for $ \Delta _3 $ and $ \Omega _2 $; in place of Eqn. \ref{eq:13.1-13b} we have the formula
\begin{equation}
	\hbar \Omega _2 ' = - \mathrm{ d_{23} } \cdot \epsilon _2 \mathcal{E}_2.
\end{equation}

\bigskip\noindent
\textbf{The Third Equation.} The foregoing phase choices produce, without further phase assignment, the third and final RWA equation. Along with Eqn. \ref{eq:13.1-12}we obtain the equation
\begin{equation}
	\frac{\mathrm d}{\mathrm d t} C_3 = - i \Delta _3 C_3 + \frac{i}{2} \Omega _2 C_2.
\end{equation}
Here, with the ordering $ E_3 > E_2 > E_1 $, the diagonal element is
\begin{equation}
	\hbar \Delta _3 = E_3 - \hbar \dot{ \zeta _3 } = E_3 - E_1 - \hbar \omega _1 - \hbar \omega _2 + \hbar \Delta _1.
\end{equation}
As with the first-step equation, a suitable constant term added to the phase $ \zeta _3 $ can make $ \Omega _2 $ real and nonnegative at time $ t = 0 $.

\bigskip\noindent
\textbf{The RWA Equations.} These phase choices, together with the neglect of exponentials having arguments
\begin{align*}
	2 \omega _1 t, \qquad 2 \omega _2 t, \qquad ( \omega _1 \pm \omega _2 ) t,
\end{align*}
produce from the Schr\"odinger equation a set of three coupled equations
\begin{equation}
	\frac{ \mathrm d }{ \mathrm d t } C_n (t) = - i \sum_m W_{nm} C_m (t) \qquad \text{or}\
	\frac{ \mathrm d }{ \mathrm d t } C(t) = - i W C(t).
\end{equation}
We can \textit{supplement the detunings and Rabi frequencies that provide the elements of the matrix $ W $ with \uline{non-Hermitian} diagonal elements to model probability loss.} When the energy ordering is $ E_1 < E_2 < E_3 $ the $ 3 \cross 3 $ RWA Hamiltonian matrix $ W $is
\begin{equation}
	[W] = \frac{1}{2}
	\begin{matrix}
		2 \Delta _1 - i \Gamma _1	&	\Omega _1 *	&	0\\
		

\end{document}

